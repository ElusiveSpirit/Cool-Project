\documentclass[a4paper,12pt]{extarticle}

\usepackage[utf8]{inputenc}
\usepackage[russian]{babel}
\usepackage{graphicx}
\usepackage[left=1.5cm,right=1.5cm,top=2cm,bottom=2cm]{geometry}
\usepackage{amsmath, amsfonts, mathtools}
%\usepackage{indentfirst}
\makeatletter
\renewcommand{\@listI}{%
	\topsep=0pt }
\makeatother

\makeatletter
\let\old@itemize=\itemize
\def\itemize{\old@itemize
	\setlength{\itemsep}{0pt}
	\setlength{\parskip}{0pt}
	\setlength{\leftskip}{5pt}
}
\makeatother

\makeatletter
\let\old@enumerate=\enumerate
\def\enumerate{\old@enumerate
	\setlength{\itemsep}{0pt}
	\setlength{\parskip}{0pt}
	\setlength{\leftskip}{5pt}
}\makeatother


\begin{document}
	\title{Разработка группового проекта\\Версия 0.1}
	\date{\today}
	\maketitle

	\setlength{\parskip}{0.5ex}
	
	\section{Задание}
	Написание многопоточного чата на $\approx\!\!1000$ человек.	Возможности:
	\begin{enumerate}
		\item написание сообщений самому себе, другому пользователю или в комнату,
		\item возможность отправления статических картинок и ссылок,
		\item поддержка шифрования,
		\item хранение истории; получение истории и статистики,
		\item доставка непрочитанных сообщений при запуске,
		\item возможность пересылать сообщения,
		\item возможность узнать, прочитано сообщение или нет,
		\item передача файлов,
		\item получение статуса пользователя,
		\item mb общение p2p с шифрованием,
		\item mb возможность отмечать сообщения избранными.
	\end{enumerate}
	
	\section{Термины}
	\begin{enumerate}
		\item \em user \em~--- пользователь:
			\begin{itemize}
				\item логин
				\item user\_id
				\item пароль
			\end{itemize}
			
		\item \em room \em~--- комната:
		\begin{itemize}
			\item room\_id
			\item список пользователей~--- [user\_id]
			\item название
		\end{itemize}
		
		\item \em message \em~--- сообщение:
		\begin{itemize}
			\item user\_id отправителя
			\item room\_id получателей
			\item тема
			\item время
			\item текст
		\end{itemize}
		
		\item \em picture \em~--- картинка:
		\begin{itemize}
			\item picture\_id
			\item данные изображения
		\end{itemize}
	\end{enumerate}
	
	\section{Реализация}
	Выбрана клиент-серверная архитектура.
	
	\noindent
	Функции сервера:
	\begin{enumerate}
		\item хранение истории сообщений,
		\item загрузка и хранение картинок,
		\item пересылка сообщений от одного пользователя в комнату,
		\item доставка сообщений для появившегося в сети пользователя,
		\item ответы на запросы пользователя,
		\item периодическая проверка связи с клиентом.
	\end{enumerate}
	Функции клиента:
	\begin{enumerate}
		\item кэширование последних сообщений и картинок (для desktop- и android-приложений),
		\item отправка сообщений на сервер,
		\item отправка запросов пользователя.
	\end{enumerate}
	Методы сервера:
	\begin{enumerate}
		\item \texttt{is\_alive}
		\item \texttt{new\_msg}
		\item \texttt{send\_data}
		\item \texttt{new\_room}
	\end{enumerate}
	Методы клиента:
	\begin{enumerate}
		\item \texttt{connect}
		\item \texttt{still\_alive}
		\item \texttt{send\_msg}
		\item \texttt{send\_data}
		\item \texttt{new\_room}
		\item \texttt{get\_history}
		\item \texttt{get\_statistics}
		\item \texttt{get\_user\_status}
		\item \texttt{get\_user\_id}
		\item \texttt{get\_rooms\_id}
		\item \texttt{get\_room\_users\_id}
		\item \texttt{create\_room}
		\item \texttt{synchronize}
		\item \texttt{disconnect}
	\end{enumerate}

	\section{Некоторые детали реализации}
	Сервер работает постоянно, поддерживая соединения со всеми клиентами в сети. При подключении клиент отправляет свои данные, и после их проверки сервер подтверждает подключение и начинает синхронизацию.
	
	Когда сообщение от отправителя приходит на сервер, он отправляет его получателю с теми же параметрами.
	
	Тяжелые объекты вроде картинок, истории или статистики, можно отправлять и получать в отдельном потоке.
	
	Картинки и ссылки в сообщении можно передавать, используя теги 
	$\#\#[pict\_id = \dots]\#\#$ и $\#\#[href = \dots]\#\#$.
	
	Также нужно предусмотреть систему безопасности: загрузить картинку, историю, сообщение или статистику можно только являясь связанным с ними.
	
		
\end{document}